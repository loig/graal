\section{Définitions de base}

\frame{
    \frametitle{Graphe}

    \definition{%
        Un graphe est est un couple $\graphedef{}$ avec~:
        \begin{itemize}
            \item $\sommets{}$ un ensemble \enavant{fini} de sommets,
            \item $\arcs{} \subseteq \sommets{}\times\sommets{}$ un ensemble \enavant{fini} d'arrêtes.
        \end{itemize}%
    }

    \uncover<2->{
        \only<1-2>{%
            \exercice{%
                Soit un graphe $\graphedef{}$.
                Quel est, en fonction de $\taille{\sommets{}}$, le nombre maximum d'arrêtes de $\graphe{}$ ?%
            }%
        }

        \only<3->{%
            \proposition{%
                Soit un graphe $\graphedef{}$, on a $\taille{\arcs{}} \leq \taille{\sommets{}}^2$.
            }%
        }
    }

    \uncover<4->{%
        \only<1-4>{%
            \exercice{%
                Soit un graphe $\graphedef{}$ sans sommets isolés.
                Quel est, en fonction de $\taille{\sommets{}}$, le nombre minimum d'arrêtes de $\graphe{}$ ?%
            }%
        }

        \only<5->{%
            \proposition{%
                Soit un graphe $\graphedef{}$ sans sommets isolés, on a $\taille{\arcs{}} \geq \lceil \taille{\sommets{}}/2 \rceil$

            }%
        }
    }
}

\frame{
    \frametitle{Vocabulaire}

    \definition[Successeurs]{%
        Dans un graphe $\graphedef{}$ les successeurs d'un sommet $\sommet{}\in\sommets{}$ sont les sommets $\sommetb{}\in\sommets{}$ tels que $(\sommet{}, \sommetb{})\in\arcs{}$.
    }

    \definition[Prédecesseurs]{%
        Dans un graphe $\graphedef{}$ les prédécesseurs d'un sommet $\sommet{}\in\sommets{}$ sont les sommets $\sommetb{}\in\sommets{}$ tels que $(\sommetb{}, \sommet{})\in\arcs{}$.
    }

    \definition[Graphe complet]{%
        Un graphe $\graphedef{}$ est dit complet si $\arcs{} = \sommets{}\times\sommets{}$.
    }

}

\frame{
    \frametitle{Dessiner les graphes}

    \begin{figure}
        \begin{tikzpicture}
            \node (a) at (0,0) {$\bullet$};
            \node (b) at (3,0) {$\bullet$};
            \node (c) at (0,3) {$\bullet$};
            \node (d) at (3,3) {$\bullet$};

            \shiftdraw{a}{b}{0}{1.5pt};
            \shiftdraw{a}{c}{-1.5pt}{0};
            \shiftdraw{a}{d}{-1pt}{1pt};
            \draw[-latex] (a) to[in=270, out=180, looseness=5] (a);

            \shiftdraw{b}{a}{0}{-1.5pt};
            \shiftdraw{b}{c}{1pt}{1pt};
            \shiftdraw{b}{d}{1.5pt}{0};
            \draw[-latex] (b) to[in=0, out=270, looseness=5] (b);

            \shiftdraw{c}{a}{1.5pt}{0};
            \shiftdraw{c}{b}{-1pt}{-1pt};
            \shiftdraw{c}{d}{0}{1.5pt};
            \draw[-latex] (c) to[in=180, out=90, looseness=5] (c);

            \shiftdraw{d}{a}{1pt}{-1pt};
            \shiftdraw{d}{b}{-1.5pt}{0};
            \shiftdraw{d}{c}{0}{-1.5pt};
            \draw[-latex] (d) to[in=90, out=0, looseness=5] (d);
        \end{tikzpicture}
        \caption{Un graphe complet à 4 sommets}
    \end{figure}
}

\frame{
    \frametitle{Les graphes vus comme des fonctions}

    \definition[À partir des successeurs]{%
        \begin{eqnarray*}
            \graphefunc{} & : & \sommets{} \rightarrow 2^\sommets{} \\
                          &   & \sommet{} \mapsto \{ \sommetb{}~|~(\sommet{}, \sommetb{})\in \arcs{} \}
        \end{eqnarray*}
    }

    \definition[À partir des prédecesseurs]{%
        \begin{eqnarray*}
            \graphefunc{} & : & \sommets{} \rightarrow 2^\sommets{} \\
                          &   & \sommet{} \mapsto \{ \sommetb{}~|~(\sommetb{}, \sommet{})\in \arcs{} \}
        \end{eqnarray*}
    }
}

\frame{
    \frametitle{Les graphes vus comme des matrices}

    \definition[matrice d'adjacence]{%
        Soit $\graphedef{}$ un graphe.
        On pose $n=\taille{\sommets{}}$ et $\sommets{}=\{\sommet{}_1, \dots, \sommet{}_n\}$.
        La matrice d'adjacence de $\graphe{}$ est $\adjmat{\graphe{}}=(\adjmatelem{})_{1\leq i, j \leq n}$ telle que~:
        \begin{eqnarray*}
            \adjmatelem{} = 1 & \textrm{si} & (\sommet{}_i, \sommet{}_j)\in\arcs{} \\
            \adjmatelem{} = 0 & \textrm{sinon} &
        \end{eqnarray*}
    }

    \pause

    \exercice{%
        Donner la matrice d'adjacence du graphe suivant~:

        \begin{center}
        \scalebox{0.6}{%
            \begin{tikzpicture}
                \node (a) at (0,0) {$\sommet{}_3$};
                \node (b) at (3,0) {$\sommet{}_4$};
                \node (c) at (0,3) {$\sommet{}_1$};
                \node (d) at (3,3) {$\sommet{}_2$};
    
                \draw[-latex] (a) to[in=270, out=180, looseness=5] (a);
    
                \shiftdraw{b}{a}{0}{-1.5pt};
    
                \shiftdraw{c}{a}{1.5pt}{0};
                \shiftdraw{c}{d}{0}{1.5pt};
    
                \shiftdraw{d}{b}{-1.5pt}{0};
                \shiftdraw{d}{c}{0}{-1.5pt};
                \draw[-latex] (d) to[in=90, out=0, looseness=5] (d);
            \end{tikzpicture}
        }
        \end{center}
    }
}

\frame{
    \frametitle{Les graphes vus comme des matrices, suite}

    \notation{%
        Soit $\graphedef{}$ un graphe et soit $\arc{} = (\sommet{}, \sommetb{})\in\arcs{}$ une arrête, on note $\sommet{} = \arcdepuis{\arc{}}$ et $\sommetb{} = \arcvers{\arc{}}$.
    }

    \definition[matrice d'incidence]{%
        Soit $\graphedef{}$ un graphe.
        On pose $n=\taille{\sommets{}}$ et $\sommets{}=\{\sommet{}_1, \dots, \sommet{}_n\}$.
        De même, on pose $m=\taille{\arcs{}}$ et $\arcs{}=\{\arc{}_1, \dots, \arc{}_m\}$.
        La matrice d'incidence de $\graphe{}$ est la matrice $\incmat{\graphe{}} = (\incmatelem{})_{1\leq i\leq n, 1\leq j\leq m}$ telle que~:
        \begin{eqnarray*}
            \incmatelem{} = 1 & \textrm{si} & \arcdepuis{\arc{}_j} = \sommet{}_i \\
            \incmatelem{} = -1 & \textrm{si} & \arcvers{\arc{}_j} = \sommet{}_i \\
            \incmatelem{} = 0 & sinon &
        \end{eqnarray*}
    }
}

\frame{
    \frametitle{Les graphes vus comme des matrices, suite, suite}

    \exercice{%
        Donner la matrice d'incidence du graphe suivant~:
        \begin{center}
            \scalebox{0.8}{%
                \begin{tikzpicture}
                    \node (a) at (0,0) {$\sommet{}_3$};
                    \node (b) at (3,0) {$\sommet{}_4$};
                    \node (c) at (0,3) {$\sommet{}_1$};
                    \node (d) at (3,3) {$\sommet{}_2$};
        
                    \node (tmp) at (1.5,3.3) {$\arc{}_1$};
                    \node (tmp) at (1.5,2.7) {$\arc{}_2$};
                    \node (tmp) at (-0.3,1.5) {$\arc{}_3$};
                    \node (tmp) at (3.3,1.5) {$\arc{}_4$};
                    \node (tmp) at (1.5,-0.3) {$\arc{}_5$};

                    \shiftdraw{b}{a}{0}{-1.5pt};
        
                    \shiftdraw{c}{a}{1.5pt}{0};
                    \shiftdraw{c}{d}{0}{1.5pt};
        
                    \shiftdraw{d}{b}{-1.5pt}{0};
                    \shiftdraw{d}{c}{0}{-1.5pt};
                \end{tikzpicture}
            }
        \end{center}
    }

    \pause

    \remarque{%
        La représentation en matrice d'incidence ne permet pas d'avoir des arcs $\arc{}$ tels que $\arcdepuis{\arc{}} = \arcvers{\arc{}}$.
    }
}

\frame{
    \frametitle{Structures de données classiques}

    Les structures de données qu'on utilise le plus souvent pour représenter les graphes sont les suivantes~:
    \begin{itemize}
        \item matrice d'adjacence
        \item listes de successeurs
        \item listes de prédécesseurs
    \end{itemize}

    \pause

    \exercice{%
        Donner la représentation en listes de successeurs et en listes de prédecesseurs du graphe dont on a précédement donné la matrice d'adjacence.
    }

    \exercice{%
        Proposer un algorithme qui permet d'obtenir une représentation en listes de successeurs à partir d'une représentation en matrice d'adjacence.
    }
}

\frame{
    \frametitle{Voisins}

    \definition[prédecesseurs]{%
        Dans un graphe $\graphedef{}$ l'ensemble des prédécesseurs d'un sommet $\sommet{}\in\sommets{}$ est noté~:
        $$\pre{\sommet{}} = \{\sommetb{}~|~(\sommetb{},\sommet{})\in\arcs{}\}$$
    }

    \definition[successeurs]{%
        Dans un graphe $\graphedef{}$ l'ensemble des successeurs d'un sommet $\sommet{}\in\sommets{}$ est noté~:
        $$\suc{\sommet{}} = \{\sommetb{}~|~(\sommet{},\sommetb{})\in\arcs{}\}$$
    }

    \definition[voisins d'un sommet]{%
        Soit un graphe $\graphedef{}$, soit un sommet $\sommet{}\in\sommets{}$, on appelle voisins de $\sommet{}$ les sommets qui sont soit des prédécesseurs soit des successeurs de $\sommet{}$, c'est-à-dire les sommets de l'ensemble $\voisins{\sommet{}} = \pre{\sommet{}}\cup\suc{\sommet{}}$.
    }
}

\frame{
    \frametitle{Degré}

    \definition[degré entrant]{%
        Soit un graphe $\graphedef{}$, soit un sommet $\sommet{}\in\sommets{}$, le degré entrant de $\sommet{}$ est $\dege{\sommet{}} = \taille{\pre{\sommet{}}}$.
    }

    \definition[degré sortant]{%
        Soit un graphe $\graphedef{}$, soit un sommet $\sommet{}\in\sommets{}$, le degré sortant de $\sommet{}$ est $\degs{\sommet{}} = \taille{\suc{\sommet{}}}$.
    }

    \definition[degré d'un sommet]{%
        Soit un graphe $\graphedef{}$, soit un sommet $\sommet{}\in\sommets{}$, le degré de $\sommet{}$ est $\degre{\sommet{}} = \dege{\sommet{}} + \degs{\sommet{}}$.
    }

    \definition[degré d'un graphe]{%
        Soit un graphe $\graphedef{}$, le degré de $\graphe{}$ est $\degre{\graphe{}} = \max_{\sommet{}\in\sommets{}} \degre{\sommet{}}$
    }
}

\frame{
    \frametitle{Graphes et relations binaires}

    On peut représenter une relation binaire $\relation{}$ sur un ensemble fini $\ensemble{}$ par un graphe $\graphedef{}$ dont
    \begin{itemize}
        \item les sommets sont les éléments de l'ensemble~: $\sommets{} = \ensemble{}$,
        \item les arrêtes représentent la relation~: $(x, y) \in \arcs{} \Leftrightarrow x\relation{}y$.
    \end{itemize}

    Un certain nombre de caractéristiques de la relation peuvent alors se repérer visuellement sur le graphe.

    \begin{columns}
        \begin{column}{5.5cm}
            \begin{block}{Réflexivité~: $\forall x, x\relation{}x$}
                \begin{center}
                    \begin{tikzpicture}
                        \node (a) at (0,0) {$x$};
                        \draw[-latex] (a) to[in=90, out=0, looseness=5] (a);
                    \end{tikzpicture}
                \end{center}
            \end{block}
            \begin{block}{Symétrie~: $\forall x, y, x\relation{}y \Rightarrow y\relation{}x$}
                \begin{center}
                    \begin{tikzpicture}
                        \node (a) at (0,0) {$x$};
                        \node (b) at (2,0) {$y$};
                        \shiftdraw{a}{b}{0}{1.5pt};
                        \shiftdraw{b}{a}{0}{-1.5pt};
                    \end{tikzpicture}
                \end{center}
            \end{block}
        \end{column}
        \begin{column}{5.5cm}
            \begin{block}{Transitivité~: $\forall x, y, z, x\relation{}y \textrm{ et } y\relation{}z \Rightarrow x\relation{}z$}
                \begin{center}
                    \begin{tikzpicture}
                        \node (a) at (0,0) {$x$};
                        \node (b) at (2,0) {$y$};
                        \node (c) at (4,0) {$z$};
                        \shiftdraw{a}{b}{0}{0};
                        \shiftdraw{b}{c}{0}{0};
                        \draw[-latex] (a) to[out=45, in=135] (c);
                    \end{tikzpicture}
                \end{center}
            \end{block}
        \end{column}
    \end{columns}
}

\frame{
    \frametitle{Graphes et relations binaires, suite}

    \exercice[relation d'équivalence]{%
        Comment peut-on caractériser le graphe représentant une relation d'équivalence (réflexive, symétrique et transitive) ?
    }

    \exercice[antisymmétrie]{%
        Une relation $\relation{}$ est dite antisymétrique si $\forall x, y, x\relation{}y \textrm{ et } y\relation{}x \Rightarrow x = y$.
        Comment cela se traduit-il visuellement dans un graphe ?
    }

    \exercice[relation d'ordre]{%
        Coment peut-on caractériser le graphe représentant une relation d'ordre (réflexive, antisymétrique et transitive)
    }
}