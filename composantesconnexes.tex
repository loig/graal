\section{Circuits et composantes connexes}

\subsection{Circuits}

\frame{
    \frametitle{La notion de circuit}

    \definition[circuit]{%
        Un chemin $\chemin{} = \sommet{}_1\sommet{}_2\dots{}\sommet{}_n$ d'un graphe $\graphe{}$ est un circuit si $\sommet{}_1 = \sommet{}_n$.
    }

    \remarque[ordonnancement]{%
        Notion centrale en ordonnancement~: on peut ordonnancer un ensemble de tâches à condition que le graphe des dépendances entre ces tâches ne contienne pas de circuit.
    }

    \pause

    \vfill

    Dans cette partie on s'intéresse à déterminer si un graphe contient un circuit ou non.
}

\frame{
    \frametitle{Circuits et graphe des chemins}

    \proposition{%
        Un graphe $\graphe{}$ contient un circuit si et seulement si son graphe des chemins $\graphechem{}$ contient une autoboucle (un arc $(\sommet{}, \sommet{})$).
    }

    \exercice{%
        Prouver le proposition.
    }
}

\frame{
    \frametitle{Points d'entrée/de sortie}

    \definition[point d'entrée]{%
        Un point d'entrée d'un graphe $\graphe{}$ est un sommet $\sommet{}\in\sommets{}$ qui n'a pas de prédecesseurs, c'est-à-dire tel que $\pre{\sommet{}} = \emptyset$.
    }

    \definition[point de sortie]{%
        Un point de sortie d'un graphe $\graphe{}$ est un sommet $\sommet{}\in\sommets{}$ qui n'a pas de succésseurs, c'est-à-dire tel que $\suc{\sommet{}} = \emptyset$.
    }

    \definition[bord]{%
        Le bord d'un graphe $\graphe{}$ est l'ensemble $\bord{\graphe{}}$ des points d'entrée et des points de sortie de $\graphe{}$.
    }
}

\frame{
    \frametitle{Bord d'un graphe et circuits}

    \proposition{%
        Un graphe $\graphe{}$ est sans circuits si et seulement si le graphe $\graphe{}\setminus\bord{\graphe{}}$ obtenu en retirant $\bord{\graphe{}}$ de $\graphe{}$ est sans circuits.
    }

    \exercice{%
        Prouver cette proposition.
        On pourra prouver avant cela qu'un circuit ne contient jamais de sommets de $\bord{\graphe{}}$.
    }

    \pause
    
    \proposition{%
        Soit un graphe $\graphe{}$ non vide et sans points d'entrée.
        $\graphe{}$ a un circuit.\\
        Soit un graphe $\graphe{}$ non vide et sans points de sortie.
        $\graphe{}$ a un circuit.
    }

    \exercice{%
        Prouver cette proposition.
        On pourra prouver dans un premier temps que si $\graphe{}$ a un chemin de longueur $k$ alors il a un chemin de longueur $k+1$.
    }
}

\frame{
    \frametitle{Algorithme glouton pour la détection de circuits}

    Des proposition précédente on déduit l'algorithme glouton suivant pour détecter si un graphe contient un circuit ou pas.

    \begin{center}
        \begin{minipage}{6cm}
            \begin{algorithmic}
                \In{$\graphe$}
                \State $H = \graphe{}$
                \While{$H\neq(\emptyset, \emptyset)$}
                    \State $X = \bord{H}$
                    \If{$X = \emptyset$}
                        \State $\graphe{}$ a un circuit
                    \Else
                        \State $H = H\setminus X$
                    \EndIf
                \EndWhile
                \State $\graphe{}$ n'a pas de circuit
            \end{algorithmic}
        \end{minipage}
    \end{center}

    \pause

    \exercice{%
        Prouver que cet algorithme fonctionne correctement (terminaison, validité du résultat).
    }
}

\subsection{DAG}

\frame{
    \frametitle{Directed Acyclic Graphs}

    \definition[DAG]{%
        Un graphe qui ne contient pas de circuits est appelé un DAG (Directed Acyclic Graph).
    }

    \definition[Tri topologique]{%
        Un tri topologique sur un graphe $\graphedef{}$ est un ordre total $\tritopo{}$ sur les sommets de $\graphe{}$ tel que $\forall (\sommet{}, \sommetb{})\in \arcs{}, \sommet{}\tritopo{}\sommetb{}$.
    }

    \pause

    \proposition{%
        Un graphe $\graphe{}$ est un DAG si et seulement si il admet un tri topologique.
    }

    \exercice{%
        Prouver la proposition.
    }
}

\frame{
    \frametitle{Alogrithme de Kahn}

    L'algorithme de Kahn permet de construire un tri topologique sur un DAG (on peut aussi l'appliquer à un graphe quelconque et vérifier si c'est un DAG~: c'est le cas si et seulement si il ne reste pas d'arcs dans le graphe à la fin de l'algorithme).

    \begin{center}
        \begin{minipage}{10cm}
            \begin{algorithmic}
                \In{$\graphe$}
                \Out{$L$ les sommets de $\graphe{}$ dans un ordre topologique}
                \State $L = \emptyset$ 
                \State $PE$ ensemble des points d'entrée de $\graphe{}$
                \While{$PE \neq \emptyset$}
                    \State enlever $\sommet{}$ de $PE$
                    \State ajouter $\sommet{}$ à la fin de $L$
                    \ForAll{$\arc{} = (\sommet{}, \sommetb{})\in \arcs{}$}
                        \State $\arcs{} = \arcs{}\setminus\{\arc{}\}$
                        \If{$\dege{\sommetb{}} = 0$}
                            \State $PE = PE\cup \{\sommetb{}\}$
                        \EndIf
                    \EndFor
                \EndWhile
            \end{algorithmic}
        \end{minipage}
    \end{center}
}

\frame{
    \frametitle{Algorithme de Kahn, suite}

    \proposition{%
        L'algorithme de Kahn termine.
    }

    \proposition{%
        À la fin de l'algorithme $\arcs{}$ est vide si et seulement si $\graphe{}$ est un DAG.
    }

    \proposition{%
        Si $\graphe{}$ est un DAG, alors à la fin de l'algorithme les sommets de $L$ sont dans un ordre topologique.
    }

    \exercice{%
        Prouver ces propositions.
    }
}

\subsection{Composantes connexes}

\frame{
    \frametitle{Definitions de base}

    \definition{%
        Soit $\eqcfc{}$ la relation sur les sommets d'un graphe $\graphe{}$ telle que $\sommet{}\eqcfc{}\sommetb{}$ si et seulement si $\sommet{}\cheminvers{}\sommetb{}$ et $\sommetb{}\cheminvers{}\sommet{}$.
    }

    \proposition{%
        $\eqcfc{}$ est une relation d'équivalence.
    }

    \exercice{%
        Le prouver.
    }

    \pause

    \definition[composante fortement connexe (SCC)]{%
        Une composante fortement connexe d'un graphe $\graphe{}$ est une classe d'équivalence de la relation $\eqcfc{}$.
    }
}

\frame{
    \frametitle{Un exemple}

    \exercice{%
        Donner les composantes fortement connexes du graphe suivant~:

        \begin{center}
            \begin{tikzpicture}
                \node (v3) at (0,0) {$\sommet{}_5$};
                \node (v4) at (3,0) {$\sommet{}_6$};
                \node (v1) at (0,3) {$\sommet{}_1$};
                \node (v2) at (3,3) {$\sommet{}_2$};
                \node (v5) at (6,0) {$\sommet{}_7$};
                \node (v6) at (6,3) {$\sommet{}_3$};
                \node (v7) at (9,3) {$\sommet{}_4$};
                \node (v8) at (9,0) {$\sommet{}_8$};

                \shiftdraw{v1}{v2}{0}{0};
                \shiftdraw{v1}{v4}{0}{0};
                \shiftdraw{v2}{v4}{0}{0};
                \shiftdraw{v3}{v1}{0}{0};
                \shiftdraw{v4}{v3}{0}{0};
                \shiftdraw{v5}{v4}{0}{0}; 
                \shiftdraw{v5}{v6}{0}{0};  
                \shiftdraw{v6}{v7}{0}{0};  
                \shiftdraw{v7}{v8}{0}{0}; 
                \shiftdraw{v8}{v6}{0}{0}; 
            \end{tikzpicture}
        \end{center}
    }

}

\frame{
    \frametitle{Propriété des composantes fortement connexes}

    \proposition{%
        Dans un graphe $\graphe{}$ si deux sommets $\sommet{}$ et $\sommetb{}$ sont dans une même composante fortement connexe, alors tous les chemins de $\sommet{}$ à $\sommetb{}$ passent uniquement par des sommets de cette même composante fortement connexe.
    }

    \exercice{%
        Prouver cette proposition.
    }
}

\frame{
    \frametitle{Graphe fortement connexe}

    \definition{%
        Un graphe est fortement connexe si et seulement si il contient une unique composante fortement connexe.
    }

    \proposition{%
        Un graphe $\graphe{}$ est fortement connexe si et seulement si $\graphechem{}$ est le graphe complet.
    }

    \exercice{%
        Prouver cette proposition.
    }
}

\frame{
    \frametitle{Quotient}

    \definition{%
        Pour un graphe $\graphedef{}$ on note $\graphe{}/\eqcfc{}$ le graphe tel que~:
        \begin{itemize}
            \item ses sommets sont les classes d'équivalences de $\eqcfc{}$,
            \item il y a un arc de $\sommets{}_1$ à $\sommets{}_2$ si et seulement si $\exists\sommet{}_1\in\sommets{}_1$, $\exists\sommet{}_2\in\sommets{}_2$ tels que $(\sommet{}_1, \sommet{}_2)\in\arcs{}$.
        \end{itemize}
    }

    \proposition{%
        Pour tout graphe $\graphe{}$, le graphe $\graphe{}/\eqcfc{}$ est un DAG.
    }

    \exercice{%
        Prouver la proposition.
    }
}

\frame{
    \frametitle{Algorithme de Kosaraju, schéma}

    \begin{center}
        Un algorithme qui calcule les SCC d'un graphe $\graphe{}$.
    \end{center}

    \definition[graphe transposé]{%
        Le graphe transposé $\transpose{\graphe{}} = (\sommets{},\transpose{\arcs{}})$ d'un graphe $\graphedef{}$ est le graphe tel que~:
        $$\transpose{\arcs{}} = \{(\sommet{},\sommetb{})~|~(\sommetb{},\sommet{})\in\arcs{}\}.$$
    }

    \vfill

    \begin{enumerate}
        \item Effectuer un parcours en profondeur (complet) de $\graphe{}$, à chaque fois qu'un sommet sort de la pile, le stocker dans une autre pile $P$.
        \item Tant que $P$ n'est pas vide~:
            \begin{enumerate}
                \item prendre le premier élément $\sommet{}$ de $P$
                \item s'il n'est pas marqué, effectuer un parcours de $\transpose{\graphe{}}$ depuis $\sommet{}$
                \item les sommets non marqués atteints constituent une SCC, les marquer
            \end{enumerate}
    \end{enumerate}
}

\frame{
    \frametitle{Algorithme de Kosaraju, intuition}

    \proposition{%
        À l'issue du parcours en profondeur, pour toute SCC $A$ et toute SCC $B$ si $A$ est avant $B$ dans le DAG $\graphe{}/\eqcfc$ alors il existe un sommet $\sommet{}\in A$ tel que pour tout $\sommetb{}\in B$ on a $\sommet{}$ qui est au dessus de $\sommetb{}$ dans la pile $P$.
    }

    \proposition[corollaire]{%
        Si on extrait tous les sommets de $P$ dans l'ordre, en ne gardant que le premier représentant rencontré de chaque SCC, les SCC sont rencontrées selon un tri topologique.
    }

    \proposition{%
        $\transpose{\graphe{}}/\eqcfc{} = \transpose{\graphe{}/\eqcfc{}}$
    }

}

\frame{
    \frametitle{Algorithme de Kosaraju, intuition, suite}

    \begin{itemize}
        \item Le premier parcours en profondeur permet de retrouver les SCC dans un ordre topologique.
        \item Dans le graphe transposé cela fait qu'on traite d'abord les SCC qui n'ont pas de successeurs.
        \item Ainsi, les parcours suivants ne sortent pas de leur SCC car~:
        \begin{itemize}
            \item au début elles n'ont pas de successeur dans le graphe transposé quotienté,
            \item puis elles n'ont pas de successeur non marqué dans le graphe transposé quotienté.
        \end{itemize}
        \item On parcourt toutes les SCC car tous les sommets sont dans la pile $P$ à la fin du parcours en profondeur.
    \end{itemize}
}

\frame{
    \frametitle{Algorithme de Kosaraju (1)}

    \begin{center}
        \begin{minipage}{10cm}
            \begin{algorithmic}
                \In{$\graphedef{}$}
                \Out{$S$ l'ensemble des SCC de $\graphe{}$}
                \State $P = init()$
                \State $S = \emptyset$
                \ForAll{$\sommet{}_0\in\sommets{}$}
                    \If{$!visite(\sommet{}_0)$}
                        \State $visite(\sommet{}_0) = true$
                        \State $t = init()$
                        \While{$!t.estVide()$}
                            \State $\sommet{} = t.observer()$
                            \If{$\exists (\sommet{}, \sommetb{})\in \arcs{}$, $!visite(\sommetb{})$}
                                \State $t.ajouter(\sommetb{})$
                                \State $visite(\sommetb{}) = true$
                            \Else
                                \State $t.extraire()$
                                \State $P.ajouter(\sommet{})$
                            \EndIf
                        \EndWhile
                    \EndIf
                \EndFor
            \end{algorithmic}
        \end{minipage}
    \end{center}
}

\frame{
    \frametitle{Algorithme de Kosaraju (2)}

    \begin{center}
        \begin{minipage}{10cm}
            \begin{algorithmic}
                \While{$!P.estVide()$}
                    \State $\sommet{}_0 = P.observer()$
                    \State $P.extraire()$
                    \If{$!marque(\sommet{}_0)$}
                        \State $marque(\sommet{}_0) = true$
                        \State $t = init()$
                        \State $SCC = \{\sommet{}_0\}$
                        \While{$!t.estVide()$}
                            \State $\sommet{} = t.observer()$
                            \If{$\exists (\sommetb{}, \sommet{})\in \arcs{}$, $!marque(\sommetb{})$}
                                \State $t.ajouter(\sommetb{})$
                                \State $marque(\sommetb{}) = true$
                                \State $SCC.ajouter(\sommetb{})$
                            \Else
                                \State $t.extraire()$
                            \EndIf
                        \EndWhile
                        \State $S.ajouter(SCC)$
                    \EndIf
                \EndWhile
            \end{algorithmic}
        \end{minipage}
    \end{center}
}

\frame{
    \frametitle{Plus efficace}

    Il existe un autre algorithme, qui est basé sur un unique parcours en profondeur~: l'algorithme de Tarjan.
}