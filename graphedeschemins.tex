\section{Graphe des chemins}

\frame{
    \frametitle{Chemins}

    \definition[chemin]{%
        Soit $\graphedef{}$ un graphe.
        Un chemin (de $\sommet{}_0$ à $\sommet{}_n$) de $\graphe{}$ est une suite $\sommet{}_0\sommet{}_1 \dots \sommet{}_n$ de sommets telle que $\forall i\in[0,n[, (\sommet{}_i, \sommet{}_{i+1})\in \arcs{}$.
    }

    \notation[chemin]{%
        S'il en existe, on peut noter $\sommet{} \cheminvers \sommetb{}$ un chemin quelconque de $\sommet{}$ à $\sommetb{}$ dans un graphe $\graphe{}$.
    }

    \definition[longueur d'un chemin]{%
        Soit $\chemin{} = \sommet{}_0\sommet{}_1\dots\sommet{}_n$ un chemin du graphe $\graphe$.
        La longueur de $\chemin{}$ est le nombre d'arrêtes de $\chemin{}$~:
        $$\taille{\chemin{}} = \taille{\{(\sommet{}_0,\sommet{}_1), (\sommet{}_1,\sommet{}_2), \dots, (\sommet{}_{n-1},\sommet{}_n)\}} = n.$$
    }
}

\frame{
    \frametitle{Graphe des chemins}

    \definition[graphe des chemins]{%
        Soit $\graphedef{}$ un graphe, le graphe $\graphechemdef$ tel que~:
        \begin{eqnarray*}
            \sommetschem & = & \sommets \\
            \arcschem & = & \{(\sommet{}, \sommetb{})~|~\exists \textrm{ chemin de } \sommet{} \textrm{ à } \sommetb{} \textrm{ dans } \graphe{}\}
        \end{eqnarray*}
        s'appelle le graphe des chemins de $\graphe{}$.
    }

    \pause

    \exercice{%
        Donner le graphe des chemins du graphe suivant~:
        \begin{center}
            \scalebox{0.6}{
            \begin{tikzpicture}
                \node (a) at (0,0) {$\bullet$};
                \node (b) at (3,0) {$\bullet$};
                \node (c) at (0,3) {$\bullet$};
                \node (d) at (3,3) {$\bullet$};

                \shiftdraw{a}{b}{0}{1.5pt};
                \draw[-latex] (a) to[in=270, out=180, looseness=5] (a);

                \shiftdraw{b}{d}{1.5pt}{0};

                \shiftdraw{c}{d}{0}{1.5pt};

                \shiftdraw{d}{c}{0}{-1.5pt};
            \end{tikzpicture}
            }
        \end{center}
    }
}

\frame{
    \frametitle{Objectif de cette partie}

    On se donne un graphe $\graphe$ et on souhaite savoir calculer (efficacement) son graphe des chemins $\graphechem$.
}

\frame{
    \frametitle{Propriétés du graphe des chemins}

    \proposition[$\graphechem$ est transitif]{%
        Soit $\graphedef{}$ un graphe et soit $\graphechemdef$ son graphe des chemins.
        Quels que soient $\sommet{}, \sommetb{}, \sommetc{} \in \sommetschem{}$ on a $$(\sommet{}, \sommetb{}) \in \arcschem{} \textrm{ et } (\sommetb{}, \sommetc{}) \in \arcschem \Rightarrow (\sommet{}, \sommetc{}) \in \arcschem{}.$$
    }

    \exercice{%
        Prouver la proposition.
    }
}

\frame{
    \frametitle{Propriétés du graphe des chemins, suite}

    \definition{%
        Soient $\graphe{}_1$ et $\graphe{}_2$ des graphes, on note $\graphe{}_1\infgraphe{}\graphe{}_2$ si $\sommets{}_1 \subseteq \sommets{}_2$ et $\arcs{}_1\subseteq\arcs{}_2$.
    }

    \pause

    \begin{columns}
        \begin{column}{5.5cm}
            \proposition{%
                $\infgraphe{}$ est une relation d'ordre.
            }
        \end{column}
        \begin{column}{5.5cm}
            \exercice{%
                Prouver la proposition.
            }
        \end{column}
    \end{columns}

    \pause

    \proposition{%
        Étant donné un graphe $\graphe{}$, son graphe des chemins $\graphechem{}$ est le plus petit graphe (pour la relation $\infgraphe{}$) transitif qui contient $\graphe{}$~:
        $$\forall H, \graphe{} \infgraphe{} H \textrm{ et } H \textrm{ transifif } \Rightarrow \graphechem{} \infgraphe{} H$$
    }

    \exercice{%
        Prouver la proposition.
    }
}

\subsection{Opérations sur les graphes}

\frame{
    \frametitle{Union}

    À partir de maintenant, on considère que tous les graphes ont le même ensemble $\sommets{}$ de sommets.

    \definition[union de graphes]{%
        Soient $\graphe{}_1=(\sommets{}, \arcs{}_1)$ et $\graphe{}_2=(\sommets{}, \arcs{}_2)$ deux graphes.
        L'union de $\graphe{}_1$ et $\graphe{}_2$ est le graphe $\graphe{}_1\uniong{}\graphe{}_2 = (\sommets{}, \arcs{}_1\cup\arcs{}_2)$.
    }

    \pause

    \begin{columns}
        \begin{column}{6.5cm}
            \proposition[associativité]{%
                $\graphe{}_1\uniong{}(\graphe{}_2\uniong{}\graphe{}_3) = (\graphe{}_1\uniong{}\graphe{}_2)\uniong{}\graphe{}_3$%
            }
        \end{column}
        \begin{column}{4.5cm}
            \exercice{Le prouver}
        \end{column}
    \end{columns}

    \pause

    \begin{columns}
        \begin{column}{6.5cm}
            \proposition[élément neutre]{%
                $\emptyset = (V, \emptyset), \graphe{}\uniong{}\emptyset = \emptyset\uniong{}\graphe{} = \graphe{}$ 
            }
        \end{column}
        \begin{column}{4.5cm}
            \exercice{Le prouver}
        \end{column}
    \end{columns}

    \pause

    \begin{columns}
        \begin{column}{6.5cm}
            \proposition[commutativité]{%
                $\emptyset = (V, \emptyset), \graphe{}\uniong{}\emptyset = \emptyset\uniong{}\graphe{} = \graphe{}$ 
            }
        \end{column}
        \begin{column}{4.5cm}
            \exercice{Le prouver}
        \end{column}
    \end{columns}
}

\frame{
    \frametitle{Représentation matricielle}

    \proposition{%
        Soit $\graphe{}_1$ un graphe représenté par la matrice d'adjacence $\adjmat{\graphe{}_1}=(a_{ij}^1)$ et soit $\graphe{}_2$ un graphe représenté par la matrice d'adjacence $\adjmat{\graphe{}_2} = (a_{ij}^2)$.
        Soit $\graphe{} = \graphe{}_1\uniong{}\graphe{}_2$ et soit $\adjmat{\graphe{}} = (\adjmatelem{})$ sa matrice d'adjacence.
        On a~: $$\adjmatelem{} = a_{ij}^1 \vee a_{ij}^2.$$
    }

    \exercice{Le prouver}

    \pause
    \vfill

    Conséquence~: on peut voir l'union des graphes comme une \enavant{somme de} leurs \enavant{matrices} d'adjacence.
}

\frame{
    \frametitle{Composition}

    \definition[composition de graphes]{%
    Soient $\graphe{}_1=(\sommets{}, \arcs{}_1)$ et $\graphe{}_2=(\sommets{}, \arcs{}_2)$ deux graphes.
    La composition de $\graphe{}_1$ et $\graphe{}_2$ est le graphe $\graphe{}_1\compog{}\graphe{}_2 = (\sommets{}, \arcs{})$ tel que~:
    $$E = \{(\sommet{}, \sommetb{})~|~\exists\sommetc{}, (\sommet{}, \sommetc{})\in\arcs{}_1 \textrm{ et } (\sommetc{}, \sommetb{})\in\arcs{}_2\}.$$
    }

    \pause

    \begin{columns}
        \begin{column}{6.5cm}
            \proposition[associativité]{%
                $\graphe{}_1\compog{}(\graphe{}_2\compog{}\graphe{}_3) = (\graphe{}_1\compog{}\graphe{}_2)\compog{}\graphe{}_3$%
            }
        \end{column}
        \begin{column}{4.5cm}
            \exercice{Le prouver}
        \end{column}
    \end{columns}

    \pause

    \begin{columns}
        \begin{column}{6.5cm}
            \proposition[élément neutre]{%
                $(\sommets{}, \{(\sommet{}, \sommet{})~|~ \sommet{}\in\sommets{}\})$ 
            }
        \end{column}
        \begin{column}{4.5cm}
            \exercice{Le prouver}
        \end{column}
    \end{columns}

    \pause

    \begin{columns}
        \begin{column}{6.5cm}
            \proposition[commutativité]{%
                La composition n'est pas commutative. 
            }
        \end{column}
        \begin{column}{4.5cm}
            \exercice{Le prouver}
        \end{column}
    \end{columns}
}

\frame{
    \frametitle{Composition, suite}

    \proposition[distributivité]{%
        La composition est distributive par rapport à l'union~:
        $$(\graphe{}_1\uniong{}\graphe{}_2)\compog{}\graphe{}_3 = (\graphe{}_1\compog{}\graphe{}_3)\uniong{}(\graphe{}_2\compog{}\graphe{}_3)$$
        $$\graphe{}_1\compog{}(\graphe{}_2\uniong{}\graphe{}_3) = (\graphe{}_1\compog{}\graphe{}_2)\uniong{}(\graphe{}_1\compog{}\graphe{}_3)$$
    }

    \exercice{Le prouver}
}

\frame{
    \frametitle{Représentation matricielle}

    \proposition{%
        Soit $\graphe{}_1$ un graphe représenté par la matrice d'adjacence $\adjmat{\graphe{}_1}=(a_{ij}^1)$ et soit $\graphe{}_2$ un graphe représenté par la matrice d'adjacence $\adjmat{\graphe{}_2} = (a_{ij}^2)$.
        Soit $\graphe{} = \graphe{}_1\compog{}\graphe{}_2$ et soit $\adjmat{\graphe{}} = (\adjmatelem{})$ sa matrice d'adjacence.
        On a~: $$\adjmatelem{} = \bigvee_{k=1}^{\taille{\sommets{}}} a_{ik}^1 \wedge a_{kj}^2.$$
    }

    \exercice{Le prouver}

    \pause
    \vfill

    Conséquence~: on peut voir l'union des graphes comme un \enavant{produit de} leurs \enavant{matrices} d'adjacence.
}

\subsection{Algorithme des puissances}

\frame{}

\subsection{Algorithme de Roy-Warshall}

\frame{}

\subsection{Retrouver les chemins}

\frame{}