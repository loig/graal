\section{Graphe des chemins}

\frame{
    \frametitle{Chemins}

    \definition[chemin]{%
        Soit $\graphedef{}$ un graphe.
        Un chemin (de $\sommet{}_0$ à $\sommet{}_n$) de $\graphe{}$ est une suite $\sommet{}_0\sommet{}_1 \dots \sommet{}_n$ de sommets telle que $\forall i\in[0,n[, (\sommet{}_i, \sommet{}_{i+1})\in \arcs{}$.
    }

    \notation[chemin]{%
        S'il en existe, on peut noter $\sommet{} \cheminvers \sommetb{}$ un chemin quelconque de $\sommet{}$ à $\sommetb{}$ dans un graphe $\graphe{}$.
    }

    \definition[longueur d'un chemin]{%
        Soit $\chemin{} = \sommet{}_0\sommet{}_1\dots\sommet{}_n$ un chemin du graphe $\graphe$.
        La longueur de $\chemin{}$ est le nombre d'arrêtes de $\chemin{}$~:
        $$\taille{\chemin{}} = \taille{\{(\sommet{}_0,\sommet{}_1), (\sommet{}_1,\sommet{}_2), \dots, (\sommet{}_{n-1},\sommet{}_n)\}} = n.$$
    }
}

\frame{
    \frametitle{Graphe des chemins}

    \definition[graphe des chemins]{%
        Soit $\graphedef{}$ un graphe, le graphe $\graphechemdef$ tel que~:
        \begin{eqnarray*}
            \sommetschem & = & \sommets \\
            \arcschem & = & \{(\sommet{}, \sommetb{})~|~\exists \textrm{ chemin de } \sommet{} \textrm{ à } \sommetb{} \textrm{ dans } \graphe{}\}
        \end{eqnarray*}
        s'appelle le graphe des chemins de $\graphe{}$.
    }

    \pause

    \exercice{%
        Donner le graphe des chemins du graphe suivant~:
        \begin{center}
            \scalebox{0.6}{
            \begin{tikzpicture}
                \node (a) at (0,0) {$\bullet$};
                \node (b) at (3,0) {$\bullet$};
                \node (c) at (0,3) {$\bullet$};
                \node (d) at (3,3) {$\bullet$};

                \shiftdraw{a}{b}{0}{1.5pt};
                \draw[-latex] (a) to[in=270, out=180, looseness=5] (a);

                \shiftdraw{b}{d}{1.5pt}{0};

                \shiftdraw{c}{d}{0}{1.5pt};

                \shiftdraw{d}{c}{0}{-1.5pt};
            \end{tikzpicture}
            }
        \end{center}
    }
}

\frame{
    \frametitle{Objectif de cette partie}

    On se donne un graphe $\graphe$ et on souhaite savoir calculer (efficacement) son graphe des chemins $\graphechem$.
}

\frame{
    \frametitle{Propriétés du graphe des chemins}

    \proposition[$\graphechem$ est transitif]{%
        Soit $\graphedef{}$ un graphe et soit $\graphechemdef$ son graphe des chemins.
        Quels que soient $\sommet{}, \sommetb{}, \sommetc{} \in \sommetschem{}$ on a $$(\sommet{}, \sommetb{}) \in \arcschem{} \textrm{ et } (\sommetb{}, \sommetc{}) \in \arcschem \Rightarrow (\sommet{}, \sommetc{}) \in \arcschem{}.$$
    }

    \exercice{%
        Prouver la proposition.
    }
}

\frame{
    \frametitle{Propriétés du graphe des chemins, suite}

    \definition{%
        Soient $\graphe{}_1$ et $\graphe{}_2$ des graphes, on note $\graphe{}_1\infgraphe{}\graphe{}_2$ si $\sommets{}_1 \subseteq \sommets{}_2$ et $\arcs{}_1\subseteq\arcs{}_2$.
    }

    \pause

    \begin{columns}
        \begin{column}{5.5cm}
            \proposition{%
                $\infgraphe{}$ est une relation d'ordre.
            }
        \end{column}
        \begin{column}{5.5cm}
            \exercice{%
                Prouver la proposition.
            }
        \end{column}
    \end{columns}

    \pause

    \proposition{%
        Étant donné un graphe $\graphe{}$, son graphe des chemins $\graphechem{}$ est le plus petit graphe (pour la relation $\infgraphe{}$) transitif qui contient $\graphe{}$~:
        $$\forall H, \graphe{} \infgraphe{} H \textrm{ et } H \textrm{ transifif } \Rightarrow \graphechem{} \infgraphe{} H$$
    }

    \exercice{%
        Prouver la proposition.
    }
}

\subsection{Opérations sur les graphes}

\frame{
    \frametitle{Union}

    À partir de maintenant, on considère que tous les graphes ont le même ensemble $\sommets{}$ de sommets.

    \definition[union de graphes]{%
        Soient $\graphe{}_1=(\sommets{}, \arcs{}_1)$ et $\graphe{}_2=(\sommets{}, \arcs{}_2)$ deux graphes.
        L'union de $\graphe{}_1$ et $\graphe{}_2$ est le graphe $\graphe{}_1\uniong{}\graphe{}_2 = (\sommets{}, \arcs{}_1\cup\arcs{}_2)$.
    }

    \pause

    \begin{columns}
        \begin{column}{6.5cm}
            \proposition[associativité]{%
                $\graphe{}_1\uniong{}(\graphe{}_2\uniong{}\graphe{}_3) = (\graphe{}_1\uniong{}\graphe{}_2)\uniong{}\graphe{}_3$%
            }
        \end{column}
        \begin{column}{4.5cm}
            \exercice{Le prouver}
        \end{column}
    \end{columns}

    \pause

    \begin{columns}
        \begin{column}{6.5cm}
            \proposition[élément neutre]{%
                $\emptyset = (V, \emptyset), \graphe{}\uniong{}\emptyset = \emptyset\uniong{}\graphe{} = \graphe{}$ 
            }
        \end{column}
        \begin{column}{4.5cm}
            \exercice{Le prouver}
        \end{column}
    \end{columns}

    \pause

    \begin{columns}
        \begin{column}{6.5cm}
            \proposition[commutativité]{%
                $\emptyset = (V, \emptyset), \graphe{}\uniong{}\emptyset = \emptyset\uniong{}\graphe{} = \graphe{}$ 
            }
        \end{column}
        \begin{column}{4.5cm}
            \exercice{Le prouver}
        \end{column}
    \end{columns}
}

\frame{
    \frametitle{Représentation matricielle}

    \proposition{%
        Soit $\graphe{}_1$ un graphe représenté par la matrice d'adjacence $\adjmat{\graphe{}_1}=(a_{ij}^1)$ et soit $\graphe{}_2$ un graphe représenté par la matrice d'adjacence $\adjmat{\graphe{}_2} = (a_{ij}^2)$.
        Soit $\graphe{} = \graphe{}_1\uniong{}\graphe{}_2$ et soit $\adjmat{\graphe{}} = (\adjmatelem{})$ sa matrice d'adjacence.
        On a~: $$\adjmatelem{} = a_{ij}^1 \vee a_{ij}^2.$$
    }

    \exercice{Le prouver}

    \pause
    \vfill

    Conséquence~: on peut voir l'union des graphes comme une \enavant{somme de} leurs \enavant{matrices} d'adjacence.
}

\frame{
    \frametitle{Algorithme pour calculer une union de graphes}

    \begin{center}
        \begin{minipage}{8cm}
            \begin{algorithmic}
                \In{$\graphe{}_1 = (\sommets{}, \arcs{}_1)$, $\graphe{}_2 = (\sommets{}, \arcs{}_2)$}
                \Out{$\graphedef{} = \graphe{}_1\uniong\graphe{}_2$}
                \State $\arcs{} = \emptyset$
                \ForAll{$\sommet{} \in \sommets{}$} 
                    \ForAll{$\sommetb{} \in \sommets{}$}
                        \If{$(\sommet{}, \sommetb{})\in \arcs{}_1$ {\bf or} $(\sommet{}, \sommetb{})\in \arcs{}_2$}
                            \State $\arcs{} = \arcs{}\cup \{(\sommet{}, \sommetb{})\}$
                        \EndIf
                    \EndFor
                \EndFor 
            \end{algorithmic}
        \end{minipage}
    \end{center}

    \vspace{1cm}
    \pause

    \exercice{%
        Donner la complexité de cet algorithme.
    }
}

\frame{
    \frametitle{Composition}

    \definition[composition de graphes]{%
    Soient $\graphe{}_1=(\sommets{}, \arcs{}_1)$ et $\graphe{}_2=(\sommets{}, \arcs{}_2)$ deux graphes.
    La composition de $\graphe{}_1$ et $\graphe{}_2$ est le graphe $\graphe{}_1\compog{}\graphe{}_2 = (\sommets{}, \arcs{})$ tel que~:
    $$E = \{(\sommet{}, \sommetb{})~|~\exists\sommetc{}, (\sommet{}, \sommetc{})\in\arcs{}_1 \textrm{ et } (\sommetc{}, \sommetb{})\in\arcs{}_2\}.$$
    }

    \pause

    \begin{columns}
        \begin{column}{6.5cm}
            \proposition[associativité]{%
                $\graphe{}_1\compog{}(\graphe{}_2\compog{}\graphe{}_3) = (\graphe{}_1\compog{}\graphe{}_2)\compog{}\graphe{}_3$%
            }
        \end{column}
        \begin{column}{4.5cm}
            \exercice{Le prouver}
        \end{column}
    \end{columns}

    \pause

    \begin{columns}
        \begin{column}{6.5cm}
            \proposition[élément neutre]{%
                $(\sommets{}, \{(\sommet{}, \sommet{})~|~ \sommet{}\in\sommets{}\})$ 
            }
        \end{column}
        \begin{column}{4.5cm}
            \exercice{Le prouver}
        \end{column}
    \end{columns}

    \pause

    \begin{columns}
        \begin{column}{6.5cm}
            \proposition[commutativité]{%
                La composition n'est pas commutative. 
            }
        \end{column}
        \begin{column}{4.5cm}
            \exercice{Le prouver}
        \end{column}
    \end{columns}
}

\frame{
    \frametitle{Composition, suite}

    \proposition[distributivité]{%
        La composition est distributive par rapport à l'union~:
        $$(\graphe{}_1\uniong{}\graphe{}_2)\compog{}\graphe{}_3 = (\graphe{}_1\compog{}\graphe{}_3)\uniong{}(\graphe{}_2\compog{}\graphe{}_3)$$
        $$\graphe{}_1\compog{}(\graphe{}_2\uniong{}\graphe{}_3) = (\graphe{}_1\compog{}\graphe{}_2)\uniong{}(\graphe{}_1\compog{}\graphe{}_3)$$
    }

    \exercice{Le prouver}
}

\frame{
    \frametitle{Représentation matricielle}

    \proposition{%
        Soit $\graphe{}_1$ un graphe représenté par la matrice d'adjacence $\adjmat{\graphe{}_1}=(a_{ij}^1)$ et soit $\graphe{}_2$ un graphe représenté par la matrice d'adjacence $\adjmat{\graphe{}_2} = (a_{ij}^2)$.
        Soit $\graphe{} = \graphe{}_1\compog{}\graphe{}_2$ et soit $\adjmat{\graphe{}} = (\adjmatelem{})$ sa matrice d'adjacence.
        On a~: $$\adjmatelem{} = \bigvee_{k=1}^{\taille{\sommets{}}} a_{ik}^1 \wedge a_{kj}^2.$$
    }

    \exercice{Le prouver}

    \pause
    \vfill

    Conséquence~: on peut voir l'union des graphes comme un \enavant{produit de} leurs \enavant{matrices} d'adjacence.
}

\frame{
    \frametitle{Algorithme pour calculer une composition de graphes}

    \begin{center}
        \begin{minipage}{9cm}
            \begin{algorithmic}
                \In{$\graphe{}_1 = (\sommets{}, \arcs{}_1)$, $\graphe{}_2 = (\sommets{}, \arcs{}_2)$}
                \Out{$\graphedef{} = \graphe{}_1\compog\graphe{}_2$}
                \State $\arcs{} = \emptyset$
                \ForAll{$\sommet{} \in \sommets{}$} 
                    \ForAll{$\sommetc{} \in \sommets{}$}
                        \ForAll{$\sommetb{} \in \sommets{}$}
                            \If{$(\sommet{}, \sommetb{})\in \arcs{}_1$ {\bf and} $(\sommetb{}, \sommetc{})\in \arcs{}_2$}
                                \State $\arcs{} = \arcs{}\cup \{(\sommet{}, \sommetc{})\}$
                            \EndIf
                        \EndFor
                    \EndFor
                \EndFor 
            \end{algorithmic}
        \end{minipage}
    \end{center}

    \pause

    \exercice{%
        Donner la complexité de cet algorithme.
    }

    \pause

    \exercice{%
        Proposer une optimisation simple permettant souvent de réduire le nombre de tours de la boucle interne.
    }
}

\subsection{Algorithme des puissances}

\frame{
    \frametitle{Exercice introductif}

    \exercice{%
        \begin{columns}
            \begin{column}{6cm}
                On considère le graphe $\graphe{}$ suivant~:
                
                ~
                
                \begin{tikzpicture}
                    \node (a) at (0,0) {$\sommet{}_3$};
                    \node (b) at (3,0) {$\sommet{}_4$};
                    \node (c) at (0,3) {$\sommet{}_1$};
                    \node (d) at (3,3) {$\sommet{}_2$};

                    \node (tmp) at (1.5, 3.3) {$\arc{}_1$};
                    \node (tmp) at (3.3, 1.5) {$\arc{}_3$};
                    \node (tmp) at (1.2, 1.8) {$\arc{}_2$};
                    \node (tmp) at (-0.7, -0.7) {$\arc{}_4$};

                    \shiftdraw{a}{d}{-1pt}{1pt};
                    \draw[-latex] (a) to[in=270, out=180, looseness=5] (a);

                    \shiftdraw{c}{d}{0}{1.5pt};

                    \shiftdraw{d}{b}{-1.5pt}{0};
                \end{tikzpicture}
            \end{column}
            \begin{column}{5cm}
                \begin{itemize}
                    \item Calculer $\graphe{}\compog{}\graphe{}$.
                    \item Quel est le lien entre les arrêtes de $\graphe{}\compog{}\graphe{}$ et les chemins de $\graphe{}$ ?
                    \item Calculer $\graphe{}\uniong{}(\graphe{}\compog{}\graphe{})$.
                    \item Quel est le lien entre les arrêtes de $\graphe{}\uniong{}(\graphe{}\compog{}\graphe{})$ et les chemins de $\graphe{}$ ?
                \end{itemize}
            \end{column}
        \end{columns}
    }
}

\frame{
    \frametitle{Une suite intéressante}

    \definition{%
        Étant donné un graphe $\graphe{}$, soit la suite $\suitepuiss{}$ définie par~:
        \begin{eqnarray*}
            \suiteelem{1} & = & \graphe{} \\
            \suiteelem{n} & = & \suiteelem{n-1}\compog{}\graphe{}
        \end{eqnarray*}
    }

    \pause

    \proposition{%
        Le n-ième terme $\suiteelem{n} = (\sommets{}, \arcs{}_n)$ de cette suite est tel que~:
        $$\arcs{}_n = \{(\sommet{}, \sommetb{})~|~\exists \textrm{ chemin de longueur } n \textrm{ de } \sommet{} \textrm{ à } \sommetb{} \textrm{ dans } \graphe{}\}$$
    }

    \exercice{%
        Prouver la proposition (on peut le faire par récurrence sur la longueur des chemins/le numéro du terme de la suite).
    }
}

\frame{
    \frametitle{Une suite intéressante, suite}

    \proposition{%
        Pour tout graphe $\graphedef{}$ on a~:
        $$\graphechem{} = \underset{n=1~~}{\overset{\taille{\arcs{}}~~}{\uniong{}}} \suiteelem{n} = \underset{n=1~~}{\overset{\taille{\sommets{}}~~}{\uniong{}}} \suiteelem{n}$$
    }

    \exercice{%
        Prouver la proposition (cela revient essentiellement à prouver que s'il y a un chemin de $\sommet{}$ à $\sommetb{}$ dans $\graphe{}$ alors il en existe un de longueur $\leq \taille{\arcs{}}$ et un de longueur $\leq \taille{\sommets{}}$).
    }

    \pause

    \begin{center}
        Cette proposition donne une méthode de calcul de $\graphechem{}$ à partir de $\graphe{}$.
    \end{center}

    \pause

    \exercice{%
        Soit un graphe $\graphedef{}$.
        Combien de compositions et d'unions fait-on pour calculer $\graphechem{}$ avec cette méthode ?
        (on peut noter $n = \min(\taille{\sommets{}}, \taille{\arcs{}})$.)%
    }
}

\frame{
    \frametitle{Une suite encore plus intéressante}

    \definition[Schéma de Hörner]{%
        Étant donné un graphe $\graphe{}$, on appelle schéma de Hörner la suite $\horner{}$ définie par~:
        \begin{eqnarray*}
            \hornerelem{1} & = & \graphe{} \\
            \hornerelem{n} & = & (\hornerelem{n-1} \compog{} \graphe{}) \uniong{} \graphe{}
        \end{eqnarray*}
    }

    \proposition{%
        $$\hornerelem{\taille{\sommets{}}} = \hornerelem{\taille{\arcs{}}} = \graphechem{}$$
    }

    \pause

    \exercice{%
        Prouver la proposition. 
        On pourra pour cela montrer par récurrence sur $n$ que $\hornerelem{n} = \uniong{}_{k=1}^n \suiteelem{k}$ pour tout $n\geq 1$.
    }
}

\frame{
    \frametitle{Une suite encore plus intéressante, suite}

    \exercice{%
        Soit un graphe $\graphedef{}$.
        Combien d'unions et de compositions fait-on pour calculer $\graphechem{}$ à l'aide du schéma de Hörner ?
    }

    \pause

    Le schéma de Hörner est donc plus efficace que le simple calcul de la suite précédente.
    On peut encore améliorer cela à partir de la proposition suivante.
    On aboutit alors à un algorithme qu'on appelle \enavant{algorithme des puissances}.

    \proposition{%
        Si $\hornerelem{p} = \hornerelem{p+1}$ alors $\forall q \geq p, \hornerelem{q} = \hornerelem{p}$.
    }

    \pause

    \exercice{%
        Prouver la proposition (par récurrence).
    }
}

\frame{
    \frametitle{Algorithme des puissances}

    \begin{center}
        \begin{minipage}{6cm}
            \begin{algorithmic}
                \In{$\graphedef{}$}
                \Out{$\hornerelem{} = \graphechem{}$}
                \State $\hornerelem{} = \graphe{}$
                \State $\hornerelem{t} = (\sommets{}, \emptyset)$
                \While{$\hornerelem{} \neq \hornerelem{t}$} 
                    \State $\hornerelem{t} = \hornerelem{}$
                    \State $\hornerelem{} = (\hornerelem{} \compog{} \graphe{}) \uniong{} \graphe{}$
                \EndWhile
            \end{algorithmic}
        \end{minipage}
    \end{center}

    \exercice{%
        Quelle est la complexité de l'algorithme des puissances en fonction de $n=\min(\taille{\sommets{}}, \taille{\arcs{}})$
    }
}

\frame{
    \frametitle{Algorithme des puissances, suite}

    \remarque{%
        Le véritable temps d'exécution de cet algorithme varie fortement en fonction du graphe considéré~: il peut s'arrêter très vite ou, au contraire, effectuer son maximum de $\min(\taille{\sommets{}}, \taille{\arcs{}})$ itérations.
    }

    \exercice{%
        Proposer un graphe (avec un ensemble de sommets et un ensemble d'arcs non vides) $\graphe{}_{min}$ tel qu'on ne fasse qu'une seule itération de la boucle de l'algorithme des puissances pour calculer son graphe des chemins.
    }

    \exercice{%
        Proposer un graphe $\graphe{}_{max}$ tel qu'on atteigne les $\min(\taille{\sommets{}}, \taille{\arcs{}})$ itérations pour calculer le graphe des chemins de $\graphe{}_{max}$ avec l'algorithme des puissances.
    }
}

\subsection{Algorithme de Roy-Warshall}

\frame{
    \frametitle{Idée générale}

    On va présenter ici un algorithme dont la complexité est meilleure que celle de l'algorithme des puissances~: de l'ordre de $\taille{\sommets{}}^3$.

    \pause

    \vfill{}

    Pour toute cette partie on travail sur un graphe $\graphedef$ et on ordonne (arbitrairement) les sommets de $\sommets{}$.

    \notation{%
        On pose $\sommets{} = \{\sommet{}_1, \sommet{}_2, \dots, \sommet{}_n\}$ et $\sommet{}_1 < \sommet{}_2 < \dots < \sommet{}_n$.%
    }

    \notation{%
        On note $\sommets{}_i = \{\sommet{}_1, \sommet{}_2, \dots, \sommet{}_i\}$, l'ensemble des sommets qui arrivent avant $\sommet{}_i$ selon l'ordre choisi.%
    }

    \proposition{%
        $$\emptyset \subset \sommets{}_0 \subset \sommets{}_1 \subset \dots \subset \sommets{}_n = \sommets{}$$%
    }
}

\frame{
    \frametitle{Idée générale, suite}

    On s'intéresse aux chemins constitués uniquement des sommets d'un sous ensemble $\sommets{}_i$ donné, au lieu de s'intéresser aux chemins d'une longueur donnée comme dans l'algorithme des puissances.

    \notation{%
        \begin{eqnarray*}
            \roy{i} = \{(\sommet{}, \sommetb{}) & | & \exists \sommet\sommet{}_{i_1}\sommet{}_{i_2}\dots\sommet{}_{i_k}\sommetb{} \textrm{ un chemin de } \graphe{} \\
                                               &   & \textrm{ tel que } \{\sommet{}_{i_1}, \sommet{}_{i_2}, \dots, \sommet{}_{i_k}\} \in \sommets{}_i\}
        \end{eqnarray*}%
    }

    \pause

    \proposition{%
        $$\forall i\geq 0, \roy{i} \subseteq \roy{i+1}$$
    }

    \exercice{%
        Prouver la proposition.
    }
}

\frame{
    \frametitle{Idée générale, suite, suite}

    \proposition{%
        $$\forall i\geq 0, \arcs{} \subseteq \roy{i}$$
    }

    \exercice{%
        Prouver la proposition.
    }

    \pause

    \proposition{%
        $$\graphechem{} = (\sommets{}, \roy{\taille{\sommets{}}}))$$
    }

    \exercice{%
        Prouver la proposition.
    }
}

\frame{
    \frametitle{Propriété fondamentale}

    \proposition{%
        $$\forall i\geq 0, \roy{i+1} = \roy{i} \cup \{ (\sommet{}, \sommetb{}) ~|~ (\sommet{}, \sommet{}_{i+1}) \in \roy{i} \textrm{ et} (\sommet{}_{i+1}, \sommetb{}) \in \roy{i}\}$$
    }

    \pause

    Pour faire la preuve on passe par le lemme suivant.

    \definition[chemin élémentaire]{%
        Soit $\chemin{} = \sommet{}_1\sommet{}_2\dots\sommet{}_n$ un chemin d'un graphe $\graphe{}$.
        Si $\forall (i,j) \neq (1,n), \sommet{}_i\neq\sommet{}_j$ alors $\chemin{}$ est dit élémentaire.
    }

    \proposition[lemme pour la preuve de la proposition précédente]{%
        Soit $\graphe{}$ un graphe, s'il existe un chemin de $\sommet{}$ à $\sommetb{}$ dans $\graphe{}$ alors il existe un chemin élémentaire qui n'utilise pas plus de sommets de $\sommet{}$ à $\sommetb{}$ dans $\graphe{}$.
    }

    \pause

    \exercice{%
        Prouver le lemme.
    }
}

\frame{
    \frametitle{Vers l'algorithme de Roy-Warshall}

    \exercice{%
        À l'aide du lemme, prouver la proposition précédente puis déduire de celle-ci un algorithme pour calculer $\graphechem$ à partir des $\roy{i}$.
    }

    \pause

    \vfill{}

    \begin{center}
        \begin{minipage}{9cm}
            \begin{algorithmic}
                \In{$\graphedef{}$}
                \Out{$(\sommets{}, \roy{\taille{\sommets{}}}) = \graphechem{}$}
                \State $\roy{0} = \arcs{}$
                \For{$i = 1$ {\bf to} $\taille{\sommets}$}
                    \State $\roy{i} = \roy{i-1}$
                    \ForAll{$\sommet{} \in \sommets{}$}
                        \ForAll{$\sommetb{} \in \sommets{}$}
                            \If{$(\sommet{}, \sommet{}_i) \in \roy{i-1}$ {\bf and} $(\sommet{}_i, \sommetb{}) \in \roy{i-1}$}
                                \State $\roy{i} = \roy{i} \cup \{(\sommet{}, \sommetb{})\}$
                            \EndIf
                        \EndFor
                    \EndFor
                \EndFor
            \end{algorithmic}
        \end{minipage}
    \end{center}
}

\frame{
    \frametitle{Algorithme de Roy-Warshall}

    \remarque{%
        Dans l'algorithme précédent il n'est pas nécessaire de faire une distinction entre $\roy{i}$ et $\roy{i-1}$~: on pourrait simplement vérifier que $(\sommet{}, \sommet{}_i)\in \roy{i}$ et $(\sommet{}_i, \sommetb{})\in \roy{i}$.
    }

    \pause

    On déduit de cette remarque l'algorithme de Roy-Warshall~:

    \begin{center}
        \begin{minipage}{9cm}
            \begin{algorithmic}
                \In{$\graphedef{}$}
                \Out{$\graphechemdef{} = (\sommets{}, \arcschem{})$}
                \State $\arcschem{} = \arcs{}$
                \ForAll{$\sommet{}_i \in \sommets{}$}
                    \ForAll{$\sommet{} \in \sommets{}$}
                        \ForAll{$\sommetb{} \in \sommets{}$}
                            \If{$(\sommet{}, \sommet{}_i) \in \arcschem{}$ {\bf and} $(\sommet{}_i, \sommetb{}) \in \arcschem{}$}
                                \State $\arcschem{} = \arcschem{} \cup \{(\sommet{}, \sommetb{})\}$
                            \EndIf
                        \EndFor
                    \EndFor
                \EndFor
            \end{algorithmic}
        \end{minipage}
    \end{center}
}

\subsection{Retrouver les chemins}

\frame{
    \frametitle{Retour sur Roy-Warshall}

    \begin{center}
        \begin{minipage}{12cm}
            \begin{algorithmic}
                \In{$\graphedef{}$}
                \Out{$\chemin{}$ tel que $\chemin{}(\sommet{}, \sommetb{})$ est un chemin de $\sommet{}$ à $\sommetb{}$ ssi il en existe}
                \ForAll{$(\sommet{}, \sommetb{})\in \arcs{}$}
                    \State $\chemin{}(\sommet{}, \sommetb{}) = (\sommet{}, \sommetb{})$
                \EndFor
                \ForAll{$\sommet{}_i \in \sommets{}$}
                    \ForAll{$\sommet{} \in \sommets{}$}
                        \ForAll{$\sommetb{} \in \sommets{}$}
                            \If{$\chemin{}(\sommet{}, \sommetb{})$ n'est pas défini}
                                \If{$\chemin{}(\sommet{}, \sommet{}_i)$ est défini {\bf and} $\chemin{}(\sommet{}_i, \sommetb{})$ est défini}
                                    \State $\chemin{}(\sommet{}, \sommetb{}) = \chemin{}(\sommet{}, \sommet{}_i).\chemin{}(\sommet_i{}, \sommetb{})$
                                \EndIf
                            \EndIf
                        \EndFor
                    \EndFor
                \EndFor
            \end{algorithmic}
        \end{minipage}
    \end{center}
}

\frame{
    \frametitle{Retour sur Roy-Warshall, propriétés}

    \proposition{%
        Si $\chemin{}(\sommet{}, \sommetb{})$ est défini alors c'est un chemin de $\sommet{}$ à $\sommetb{}$ dans $\graphe{}$.%
    }

    \exercice{%
        Prouver la proposition.%
    }

    \pause

    \proposition{%
        S'il existe un chemin de $\sommet{}$ à $\sommetb{}$ alors $\chemin{}(\sommet{}, \sommetb{})$ est défini.
    }

    \exercice{%
        Prouver la proposition.%
    }
}

\frame{
    \frametitle{Tables de routage}

    On peut stocker les chemins plus efficacement en utilisant une table de routage~:

    \begin{center}
        \begin{minipage}{12cm}
            \begin{algorithmic}
                \In{$\graphedef{}$}
                \Out{$\route{}$ tel que $\route{}(\sommet{}, \sommetb{})$ est le premier sommet par lequel on passe sur un chemin de $\sommet{}$ à $\sommetb{}$ ssi il en existe}
                \ForAll{$(\sommet{}, \sommetb{})\in \arcs{}$}
                    \State $\route{}(\sommet{}, \sommetb{}) = \sommetb{}$
                \EndFor
                \ForAll{$\sommet{}_i \in \sommets{}$}
                    \ForAll{$\sommet{} \in \sommets{}$}
                        \ForAll{$\sommetb{} \in \sommets{}$}
                            \If{$\route{}(\sommet{}, \sommetb{})$ n'est pas défini}
                                \If{$\route{}(\sommet{}, \sommet{}_i)$ est défini {\bf and} $\route{}(\sommet{}_i, \sommetb{})$ est défini}
                                    \State $\route{}(\sommet{}, \sommetb{}) = \route{}(\sommet{}, \sommet{}_i)$
                                \EndIf
                            \EndIf
                        \EndFor
                    \EndFor
                \EndFor
            \end{algorithmic}
        \end{minipage}
    \end{center}
}

\frame{
    \frametitle{Tables de routage, suite}

    \exercice{%
        \begin{columns}
            \begin{column}{6cm}
                On considère le graphe $\graphe{}$ suivant~:
                
                ~
                
                \begin{tikzpicture}
                    \node (a) at (0,0) {$\sommet{}_3$};
                    \node (b) at (3,0) {$\sommet{}_4$};
                    \node (c) at (0,3) {$\sommet{}_1$};
                    \node (d) at (3,3) {$\sommet{}_2$};

                    \node (tmp) at (1.5, 3.3) {$\arc{}_1$};
                    \node (tmp) at (3.3, 1.5) {$\arc{}_2$};
                    \node (tmp) at (1.2, 1.8) {$\arc{}_5$};
                    \node (tmp) at (3.7, 3.7) {$\arc{}_4$};
                    \node (tmp) at (1.5, -0.3) {$\arc{}_3$};

                    \shiftdraw{a}{d}{-1pt}{1pt};

                    \shiftdraw{c}{d}{0}{1.5pt};

                    \shiftdraw{d}{b}{-1.5pt}{0};
                    \draw[-latex] (d) to[in=0, out=90, looseness=5] (d);

                    \shiftdraw{b}{a}{0}{1.5pt};
                \end{tikzpicture}
            \end{column}
            \begin{column}{5cm}
                \begin{itemize}
                    \item Calculer $\route{}(\sommet{}, \sommetb{})$ pour tout $\sommet{}\in\sommets{}$ et tout $\sommetb{}\in\sommets{}$.
                    \item Le représenter comme une matrice ($\route{}(\sommet{}_i, \sommet{}_j)$ sera indiqué dans la case $(i,j)$ de la matrice)
                    \item En déduire un chemin de $\sommet{}_1$ à $\sommet{}_3$.
                \end{itemize}
            \end{column}
        \end{columns}
    }

}